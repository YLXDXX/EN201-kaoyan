\bta{2022}


\section{Use of English}

\noindent
\textbf{Directions:}\\
Read the following text. Choose the best word(s) for each numbered blank
and mark A, B, C or D on the
ANSWER SHEET. (10 points)

\TiGanSpace


The idea that plants have some degree of consciousness first took root
in the early 2000s; the term ``plant
neurobiology'' was  \cloze around the notion that some aspects of plant behavior could be
\cloze 
to intelligence in
animals.
 \cloze 
plants lack brains, the firing of electrical signals in their stems and
leaves nonetheless triggered
responses that  \cloze 
consciousness, researchers previously reported.


But such an idea is untrue, according to a new opinion article. Plant
biology is complex and fascinating, but it
 \cloze 
so greatly from that of animals that so-called
 \cloze 
 of plants' intelligence is inconclusive, the authors wrote.
 
 
Beginning in 2006, some scientists have
 \cloze 
 that plants possess neuron-like cells that interact with
hormones and neurotransmitters,
 \cloze 
``a plant nervous system,  \cloze 
to that in animals,'' said lead study author 
Lincoln Taiz, ``They  \cloze claimed that plants have `brain-like command centers' at their root
tips.''



This
 \cloze 
makes sense if you simplify the workings of a complex brain,
 \cloze 
it to an array of electrical pulses; cells in plants also communicate through electrical signals.
 \cloze , the signaling in a plant is only
 \cloze 
similar to the firing in a complex animal brain, which is more than ``a
mass of cells that communicate by
electricity,'' Taiz said.



``For consciousness to evolve, a brain with a threshold
 \cloze 
of complexity and capacity is required,'' he 
  \cloze . ``Since plants don't have nervous systems, the
 \cloze 
that they have consciousness are effectively zero.''


And what's so great about consciousness, anyway? Plants can't run away
from
 \cloze , so investing energy in
a body system which
 \cloze 
a threat and can feel pain would be a very
 \cloze 
evolutionary strategy, according to
the article.

\newpage

\begin{enumerate}
	%\renewcommand{\labelenumi}{\arabic{enumi}.}
	% A(\Alph) a(\alph) I(\Roman) i(\roman) 1(\arabic)
	%设定全局标号series=example	%引用全局变量resume=example
	%[topsep=-0.3em,parsep=-0.3em,itemsep=-0.3em,partopsep=-0.3em]
	%可使用leftmargin调整列表环境左边的空白长度 [leftmargin=0em]
	\item

\fourchoices
{coined}
{discovered}
{collected}
{issued}



\item

\fourchoices
{attributed}
{directed}
{compared}
{confined}

\item

\fourchoices
{Unless}
{When}
{Once}
{Though}


\item

\fourchoices
{cope with}
{consisted of}
{hinted at}
{extended in}


\item

\fourchoices
{suffers}
{benefits}
{develops}
{differs}


\item

\fourchoices
{acceptance}
{evidence}
{cultivation}
{creation}


\item
\fourchoices
{doubted}
{denied}
{argued}
{requested}


\item

\fourchoices
{adapting}
{forming}
{repairing}
{testing}

\item 
\fourchoices
{analogous}
{essential}
{suitable}
{sensitive}


\item

\fourchoices
{just}
{ever}
{still}
{even}



\item

\fourchoices
{restriction}
{experiment}
{perspective}
{demand}


\item

\fourchoices
{attaching}
{reducing}
{returning}
{exposing}



\item

\fourchoices
{However}
{Moreover}
{Therefore}
{Otherwise}


\item
\fourchoices
{temporarily}
{literally}
{superficially}
{imaginarily}


\item

\fourchoices
{list}
{level}
{label}
{local}


\item

\fourchoices
{recalled}
{agreed}
{questioned}
{added}


\item

\fourchoices
{chances}
{risks}
{excuses}
{assumptions}


\item

\fourchoices
{danger}
{failure}
{warning}
{control}



\item

\fourchoices
{represents}
{includes}
{reveals}
{recognizes}


\item

\fourchoices
{humble}
{poor}
{practical}
{easy}


\end{enumerate}


\vfil

\section{Reading Comprehension}


\noindent
\textbf{Part A}\\
\textbf{Directions:}\\
Read the following four texts. Answer the questions below each text by
choosing A, B, C or
D. Mark your
answers on the ANSWER SHEET. (40 points)





\newpage

\subsection{Text 1}

People often complain that plastics are too durable. Water
bottles, shopping bags, and other trash litter the
planet, from Mount Everest to the Mariana Trench, because plastics are
everywhere and don't break down easily.
But some plastic materials change over time. They crack and frizzle.
They ``weep'' out additives. They melt into
sludge. All of which creates huge headaches for institutions, such as
museums, trying to preserve culturally
important objects. The variety of plastic objects at risk is dizzying:
early radios, avant-garde sculptures, celluloid
animation stills from Disney films, the first artificial heart.


Certain artifacts are especially vulnerable because some pioneers in
plastic art didn't always know how to
mix ingredients properly, says Thea van Oosten, a polymer chemist who,
until retiring a few years ago, worked for
decades at the Cultural Heritage Agency of the Netherlands. ``It s like
baking a cake: If you don't have exact
amounts, it goes wrong.'' she says. ``The object you make is already a
time bomb.''



And sometimes, it's not the artist's fault. In the 1960 s, the Italian
artist Picro Gilardi began to create hundreds
of bright, colorful foam pieces. Those pieces included small beds of
roses and other items as well as a few dozen
``nature carpets''---large rectangles decorated with foam pumpkins,
cabbages, and watermelons. He wanted
viewers to walk around on the carpets---which meant they had to be
durable.



Unfortunately, the polyurethane foam he used is inherently unstable.
It's especially vulnerable to light
damage, and by the mid-1990s, Gilardi's pumpkins, roses, and other
figures were splitting and crumbling.
Museums locked some of them away in the dark.




So van Oosten and her colleagues worked to preserve Gilardi's
sculptures. They infused some with stabilizing
and consolidating chemicals. Van Oosten calls those chemicals
``sunscreens'' because their goal was to prevent
further light damage and rebuild worn polymer fibers. She is proud that
several sculptures have even gone on
display again, albeit sometimes beneath protective cases.




Despite success stories like van Oosten's, preservation of plastics will
likely get harder. Old objects continue
to deteriorate. Worse, biodegradable plastics designed to disintegrate,
are increasingly common.



And more is at stake here than individual objects. Joana Lia Ferreira, an
assistant professor of conservation
and restoration at the nova School of Science and Technology, notes that
archaeologists first defined the great
material ages of human history Stone Age, Iron Age, and so on after
examining artifacts in museums. We now live
in an age of plastic, she says, ``and what we decide to collect
today, what we decide to preserve\ldots will have a
strong impact on how in the future we'll be seen.''


\begin{enumerate}[resume]
	%\renewcommand{\labelenumi}{\arabic{enumi}.}
	% A(\Alph) a(\alph) I(\Roman) i(\roman) 1(\arabic)
	%设定全局标号series=example	%引用全局变量resume=example
	%[topsep=-0.3em,parsep=-0.3em,itemsep=-0.3em,partopsep=-0.3em]
	%可使用leftmargin调整列表环境左边的空白长度 [leftmargin=0em]
	\item
According to Paragraph 1, museums are faced with difficulties
in \lineread.


\fourchoices
{maintaining their plastic items}
{obtaining durable plastic artifacts}
{handling outdated plastic exhibits}
{classifying their plastic collections}



\item
Van Oosten believes that certain plastic objects are \lineread.


\fourchoices
{immune to decay}
{improperly shaped}
{inherently flawed}
{complex in structure}



\item
 Museums stopped exhibiting some of Gilardi's artworks to
\lineread.


\fourchoices
{keep them from hurting visitors}
{duplicate them for future display}
{have their ingredients analyzed}
{prevent them from further damage}



\item
The author thinks that preservation of plastics is \lineread.


\fourchoices
{costly}
{unworthy}
{unpopular}
{challenging}



\item
In Ferreira's opinion, preservation of plastic artifacts
\lineread.


\fourchoices
{will inspire future scientific research}
{has profound historical significance}
{will help us separate the material ages}
{has an impact on today's cultural life}

\end{enumerate}


\newpage
\subsection{Text 2}

As the latest crop of students pen their undergraduate application form
and weigh up their options, it may be
worth considering just how the point, purpose and value of a degree has
changed and what Generation Z need to
consider as they start the third stage of their educational journey.



Millennials were told that if you did well in school, got a decent
degree, you would be set up for life. But that
promise has been found wanting. As degrees became universal, they became
devalued. Education was no longer a
secure route of social mobility. Today, 28 per cent of graduates in the
UK are in non-graduate roles, a percentage
which is double the average among OECD countries.



This is not to say that there is no point in getting a degree, but
rather stress that a degree is not for everyone,
that the switch from classroom to lecture hall is not an inevitable one
and that other options are available.




Thankfully, there are signs that this is already happening, with
Generation Z seeking to learn from their
millennial predecessors, even if parents and teachers tend to be still
set in the degree mindset. Employers have
long seen the advantages of hiring school leavers who often prove
themselves to be more committed and loyal
employees than graduates. Many too are seeing the advantages of
scrapping a degree requirement for certain roles.




For those for whom a degree is the desired route, consider that this may
well be the first of many. In this age
of generalists, it pays to have specific knowledge or skills.
Postgraduates now earn 40 per cent more than
graduates. When more and more of us have a degree, it makes sense to
have two.




It is unlikely that Generation Z will be done with education at 18 or
21; they will need to be constantly
up-skilling throughout their career to stay employable. It has been
estimated that this generation, due to the
pressures of technology, the wish for personal fulfilment and desire for
diversity, will work for 17 different
employers over the course of their working life and have five different
careers. Education, and not just knowledge
gained on campus, will be a core part of Generation Z's career
trajectory.



Older generations often talk about their degree in the present and
personal tense: `I am a geographer.' or `I
am a classist.' Their sons or daughters would never say such a thing;
it's as if they already know that their degree
won't define them in the same way.

\begin{enumerate}[resume]
	%\renewcommand{\labelenumi}{\arabic{enumi}.}
	% A(\Alph) a(\alph) I(\Roman) i(\roman) 1(\arabic)
	%设定全局标号series=example	%引用全局变量resume=example
	%[topsep=-0.3em,parsep=-0.3em,itemsep=-0.3em,partopsep=-0.3em]
	%可使用leftmargin调整列表环境左边的空白长度 [leftmargin=0em]
	\item
 The author suggests that Generation Z should \lineread.


\fourchoices
{be careful in choosing a college}
{be diligent at each educational stage}
{reassess the necessity of college education}
{postpone their undergraduate application}


\item
The percentage of UK graduates in non-graduate roles
reflect \lineread.


\fourchoices
{Millennial's opinions about work}
{the shrinking value of a degree}
{public discontent with education}
{the desired route of social mobility}



\item
 The author considers it a good sign that \lineread.


\fourchoices
{Generation Z are seeking to earn a decent degree}
{School leavers are willing to be skilled workers}
{Employers are taking a realistic attitude to degrees}
{Parents are changing their minds about education}



\item
It is advised in Paragraph 5 that those with one degree
should \lineread.


\fourchoices
{make an early decision on their career}
{attend on the job training programs}
{team up with high-paid postgraduates}
{further their studies in a specific field}



\item
What can be concluded about Generation Z from the last two
paragraphs?


\fourchoices
{Lifelong learning will define them.}
{They will make qualified educators.}
{Degrees will no longer appeal them.}
{They will have a limited choice of jobs.}

\end{enumerate}


\newpage
\subsection{Text 3}


Enlightening, challenging, stimulating, fun. These were some of the words
that Nature readers used to
describe their experience of art-science collaborations in a series of
articles on partnerships between artists and
researchers. Nearly 40\% of the roughly 350 people who responded to an
accompanying poll said, they had
collaborated with artists, and almost all said they would consider doing
so in future.




Such an encouraging results is not surprising. Scientists are
increasingly seeking out visual artists to help
them communicate their work to new audiences. ``Artists help scientists
reach a broader audience and make
emotional connections that enhance learning.'' One respondent said.



One example of how artists and scientists have together rocked the
scenes came last month when the Sydney
Symphony Orchestra performed a reworked version of Antonio Vivaldi's The
Four Seasons. They reimagined the
300-year-old score by injecting the latest climate prediction data for
each season-provided by Monash University's
Climate Change Communication Research Hub. The performance was a
creative call to action ahead of
November's United Nations Climate Change Conference in Glasgow, UK.




But a genuine partnership must be a two-way street. Fewer artist than
scientists responded to the Nature poll,
however, several respondents noted that artists do not simply assist
scientists with their communication
requirements. Nor should their work be considered only as an object of
study. The alliances are most valuable
when scientists and artists have a shared stake in a project, are able
to jointly design it and can critique each
other's work. Such an approach can both prompt new research as well as
result in powerful art.




More than half a century ago, the Massachusetts Institute of Technology
opened its Center for Advanced
Visual Studies (CAVS) to explore the role of technology in culture. The
founders deliberately focused their
projects around light-hance the ``visual studies'' in the name. Light
was a something that both artists and scientists
had an interest in and therefore could form the basis of collaboration.
As science and technology progressed, and
divided into more sub-disciplines, the centre was simultaneously looking
to a time when leading researchers could
also be artists, writers and poets, and vice versa.



Nature's poll findings suggest that this trend is as strong as ever, but,
to make a collaboration work both sides
need to invest time and embrace surprise and challenge. The reach of
art-science tie-ups needs to go beyond the
necessary purpose of research communication, and participants. Artists
and scientists alike are immersed in
discovery and invention, and challenge and critique are core to
both, too.

\begin{enumerate}[resume]
	%\renewcommand{\labelenumi}{\arabic{enumi}.}
	% A(\Alph) a(\alph) I(\Roman) i(\roman) 1(\arabic)
	%设定全局标号series=example	%引用全局变量resume=example
	%[topsep=-0.3em,parsep=-0.3em,itemsep=-0.3em,partopsep=-0.3em]
	%可使用leftmargin调整列表环境左边的空白长度 [leftmargin=0em]
	\item
 According to paragraph 1, art-science collaborations have \lineread.


\fourchoices
{caught the attention of critics}
{received favorable responses}
{promoted academic publishing}
{sparked heated public disputes}



\item
The reworked version of The Four Seasons is mentioned to show
that \lineread.


\fourchoices
{art can offer audiences easy access to science}
{science can help with the expression of emotions}
{public participation in science has a promising future}
{art is effective in facilitating scientific innovations}



\item
 Some artists seem to worry about in the art-science
partnership \lineread.


\fourchoices
{their role may be underestimated}
{their reputation may be impaired}
{their creativity may be inhibited}
{their work may be misguided}



\item
 What does the author say about CAVS?

\fourchoices
{It was headed alternately by artists and scientists.}
{It exemplified valuable art-science alliances.}
{Its projects aimed at advancing visual studies.}
{Its founders sought to raise the status of artists.}


\item
In the last paragraph, the author holds that art-science
collaborations \lineread.

\fourchoices
{are likely to go beyond public expectations}
{will intensify interdisciplinary competition}
{should do more than communicating science}
{are becoming more popular than before}



\end{enumerate}


\newpage
\subsection{Text 4}


The personal grievance provisions of New Zealand's Employment Relations
Act 2000 (ERA) prevent an
employer from firing an employee without good cause. Instead, dismissals
must be justified. Employers must both
show cause and act in a procedurally fair way.



Personal grievance procedures were designed to guard the jobs of
ordinary workers from ``unjustified
dismissals'' The premise was that the common law of contract lacked
sufficient safeguards for workers against
arbitrary conduct by management. Long gone are the days when a boss
could simply give an employee contractual
notice.




But these provisions create difficulties for businesses when applied to
highly paid managers and executives.
As countless boards and business owners will attest, constraining firms
from firing poorly performing,
high-earning managers is a handbrake on boosting productivity and
overall performance. The difference between
C-grade and A-grade managers may very well be the difference between
business success or failure. Between
preserving the jobs of ordinary workers or losing them. Yet mediocrity
is no longer enough to justify a dismissal.
Consequently-and paradoxically - laws introduced to protect the jobs of
ordinary workers may be placing those
jobs at risk.




If not placing jobs at risk, to the extent employment protection laws
constrain business owners from
dismissing under-performing managers, those laws act as a constraint on
firm productivity and therefore on
workers' wages. Indeed, in ``An International Perspective on New
Zealand's Productivity Paradox'' (2014), the
Productivity Commission singled out the low quality of managerial
capabilities as a cause of the country's poor
productivity growth record.




Nor are highly paid managers themselves immune from the harm caused by
the ERA's unjustified dismissal
procedures. Because employment protection laws make it costlier to fire
an employee, employers are more
cautious about hiring new staff. This makes it harder for the marginal
manager to gain employment. And firms pay
staff less because firms carry the burden of the employment arrangement
going wrong.




Society also suffers from excessive employment protections. Stringent
job dismissal regulations adversely
affect productivity growth and hamper both prosperity and overall
well-being.




Across the Tasman Sea, Australia deals with the unjustified dismissal
paradox by excluding employees
earning above a specified ``high-income threshold'' from the protection
of its unfair dismissal laws. In New
Zealand, a 206 private members' Bill tried to permit firms and
high-income employees to contract out of the
unjustified dismissal regime. However, the mechanisms proposed were
unwieldy and the Bill was voted down
following the change in government later that year.



\begin{enumerate}[resume]
	%\renewcommand{\labelenumi}{\arabic{enumi}.}
	% A(\Alph) a(\alph) I(\Roman) i(\roman) 1(\arabic)
	%设定全局标号series=example	%引用全局变量resume=example
	%[topsep=-0.3em,parsep=-0.3em,itemsep=-0.3em,partopsep=-0.3em]
	%可使用leftmargin调整列表环境左边的空白长度 [leftmargin=0em]
	\item
 The personal grievance provisions of the ERA are intended to \lineread.


\fourchoices
{punish dubious corporate practices}
{improve traditional hiring procedures}
{exempt employers from certain duties}
{protect the rights of ordinary workers}



\item
 It can be learned from paragraph 3 that the provisions may \lineread.

\fourchoices
{hinder business development}
{undermine managers authority}
{affect the public image of the firms}
{worsen labor-management relations}


\item
 Which of the following measures would the Productivity Commission
support?


\fourchoices
{Imposing reasonable wage restraints.}
{Enforcing employment protection laws.}
{Limiting the powers of business owners.}
{Dismissing poorly performing managers.}


\item
 What might be an effect of ERA's unjustified dismissal procedures?


\fourchoices
{Highly paid managers lose their jobs.}
{Employees suffer from salary cuts.}
{Society sees a rise in overall well-being.}
{Employers need to hire new staff.}


\item
 It can be inferred that the ``high-income threshold'' in Australia \lineread.


\fourchoices
{has secured managers' earnings}
{has produced undesired results}
{is beneficial to business owners}
{is difficult to put into practice}


	
\end{enumerate}

\newpage

\noindent
\textbf{Part B}\\
\textbf{Directions:}\\
Read the following text and answer the questions by choosing the most
suitable subheading from the list
A-G for each numbered paragraphs (41-45). There are two extra subheadings
which you do not need to use.
Mark your answers on ANSWER SHEET. (10 points)

\TiGanSpace


\textbf{(41) Teri Byrd}

I was a zoo and wildlife park employee for years. Both the wildlife park
and zoo claimed to be operating for
the benefit of the animals and for conservation purposes. This claim was
false. Neither one of them actually
participated in any contributions whose bottom line is much more
important than the condition of the animals.
Animals despise being captives in zoos. No matter how you enhance
enclosures, they do not allow for freedom, a
natural diet or adequate time for transparency with these institutions,
and it's past time to eliminate zoos from our
culture.



\textbf{(42) Karen R. Sime}

As a zoology professor, I agree with Emma Marris that zoo displays can
be sad and cruel. But she
underestimates the educational value of zoos. The zoology program at my
university attracts students for whom
zoo visits were the crucial formative experience that led them to major
in biological sciences. These are mostly
students who had no opportunity as children to travel to wilderness
areas, wildlife refuges or national parks.
Although good TV shows can help stir children's interest in
conservation, they cannot replace the excitement of a
zoo visit as an intense, immersive and interactive experience. Surely
there must be some middle ground that
balances zoos treatment of animals with their educational potential.




\textbf{(43) Reg Newberry}


Emma Marris's article is an insult and a disservice to the thousands of
passionate who work tirelessly to
improve the lives of animals and protect our planet. She uses outdated
research and decades-old examples to
undermine the noble mission of organization committed to connecting
children to a world beyond their own. Zoos
are at the forefront of conservation and constantly evolving to improve
how thy care for animals and protect each
species in its natural habitat. Are there tragedies? Of course. But they
are the exception not the norm that Ms
Marris implies A distressed animal in a zoo will get as good or better
treatment than most of us at our local
hospital.




\textbf{(44) Dean Gallea}

As a fellow environmentalist animal-protection advocate and longtime
vegetarian. I could properly be in the
same camp as Emma Marris on the issue of zoos. But I believe that
well-run zoos and the heroic animals that
suffer their captivity so serve a higher purpose. Were it not for
opportunities to observe these beautiful wild
creatures close to home many more people would be driven by their
fascination to travel to wild areas to seek out
disturb and even hunt them down.
Zoos are in that sense similar to natural history and archeology museums
serving to satisfy our need for contact
with these living creatures while leaving the vast majority undisturbed
in their natural environments.



\textbf{(45) John Fraser}

Emma Marris selectively describes and misrepresents the findings of our
research. Our studies focused on
the impact of zoo experiences on how people think about themselves and
nature and the data points extracted from
our studies. Zoos are tools for thinking. Our research provides strong
support for the value of zoos in connecting
people with animals and with nature. Zoos provide a critical voice for
conservation and environmental protection.
They afford an opportunity for people from all backgrounds to encounter
a range of animals from drone bees to
springbok or salmon to better understand the natural world we live in.


\begin{listmatch}
\item 
Zoos, which spare no effort to take of animals, should not be
subjected to unfair criticism.


\item 
To pressure zoos to spend less on their animals would lead to
inhumane outcomes for the precious creatures in
their care.


\item 
 While animals in captivity deserve sympathy, zoos play a
significant role in starting young people down the
path of related sciences.


\item 
 Zoos save people trips to wilderness areas and thus contribute
to wildlife conservation.


\item 
For wild animals that cannot be returned to their natural
habitats, zoos offer the best alternative.


\item 
Zoos should have been closed down as they prioritize money
making over animals' wellbeing.


\item 
Marris distorts our findings which actually prove that zoos
serve as an indispensable link between man and
nature.


\end{listmatch}

\phantom{ \linefill \linefill \linefill \linefill \linefill}


\noindent
\textbf{Part C}\\
\textbf{Directions:}\\
Read the following text carefully and then translate the
underlined segments into Chinese. Your
translation should be written neatly on ANSWER SHEET. (10 points)

\TiGanSpace


\begin{center}
	\textbf{ 
	The Man Who Broke Napoleon's Codes-Mark Urban}
\end{center}


Between 1807 and 1814 the Iberian Peninsula (comprising Spain and
Portugal) was the scene of a titanic and
merciless struggle. It took place on many different planes: between
Napoleon's French army and the angry
inhabitants; between the British, ever keen to exacerbate the emperor's
difficulties, and the marshals sent from
Paris to try to keep them in check; between new forces of science and
meritocracy and old ones of conservatism
and birth. \transnum  \uline{It was also, and this is unknown even to many people
well read about the period, a battle between
those who made codes and those who broke them.}





I first discovered the Napoleonic cryptographic battle a few years ago
when I was reading Sir Charles
Oman's epic History of the Peninsular War. In volume V he had attached
an appendix, The Scovell Ciphers. \transnum  \uline{It
listed many documents in code that had been captured from the French
army of Spain, and whose secrets had been
revealed by the work of one George Scovell, an officer in British
headquarters.} Oman rated Scovell's significance
highly, but at the same time, the general nature of his History meant
that \transnum  \uline{he could not analyze carefully what
this obscure officer may or may not have contributed to that great
struggle between nations or indeed tell us
anything much about the man himself.} I was keen to read more, but was
surprised to find that Oman's appendix,
published in 1914, was the only considered thing that had been written
about this secret war.



I became convinced that this story was every bit as exciting and
significant as that of Enigma and the
breaking of German codes in the Second World War. The question was,
could it be told?




Studying Scovell's papers at the Public Record Office, London, I found
that he had left an extensive journal
and copious notes about his work in the Peninsula. What was more, many
original French dispatches had been
preserved in this collection, which I realized was priceless. \transnum  \uline{There
may have been many spies and intelligence
officers during the Napoleonic Wars, but it is usually extremely
difficult to find the material they actually
provided or worked on.}



Furthermore, Scovell's story involved much more than just intelligence
work. His status in Lord Wellington's
headquarters and the recognition given to him for his work were all
bound up with the class politics of the army at
the time. His tale of self-improvement and hard work would make a
fascinating biography in its own right, but
represents something more than that. \transnum  \uline{Just as the code breaking has
its wider relevance in the struggle for
Spain, so his attempts to make his way up the promotion ladder speak
volumes about British society.}




The story of Wellington himself also gripped me. Half a century ago his
campaigns were considered a central
part of the British historical mythology and spoon-fed to schoolboys.
More recently this has not been the case,
which is a great shame. A generation has grown up.



\newpage
\section{Writing}


\noindent
\textbf{Part A}\\
\textbf{51. Directions:}

Write an email to a professor at a British university, inviting him/her
to organize a team for the international
innovation contest to be held at your university.

You should write about 100 words on the ANSWER SHEET.

\textbf{Do not} use your own name in the email. Use ``Li Ming'' instead. (10
points)


\vspace{2em}

\noindent
\textbf{Part B}\\
\textbf{52. Directions:}

Write an essay of 160-200 words based on the following picture below. In
your essay, you should
\begin{listwrite}
\item 
 describe the picture briefly,

\item 
 explain its intended meaning and

\item 
 give your comments.
\end{listwrite}

You should write neatly on the ANSWER SHEET. (20 points)

\begin{figure}[h!]
	\centering
	\includesvg[width=0.55\linewidth]{picture/svg/picture-01}
\end{figure}



