
\bta{2003}


\section{Use of English}

\noindent
\textbf{Directions:}\\
Read the following text. Choose the best word (s) for each
	numbered blank and mark A, B, C OR D on ANSWER SHEET 1. (10 points)

\TiGanSpace


Teachers need to be aware of the emotional, intellectual, and physical
changes that young adults experience. And they also need to give serious
\cloze to how they can best \cloze such changes. Growing
bodies need movement and \cloze , but not just in ways that
emphasize competition. \cloze they are adjusting to their new
bodies and a whole host of new intellectual and emotional challenges,
teenagers are especially self-conscious and need the \cloze that
comes from achieving success and knowing that their accomplishments are
\cloze by others. However, the typical teenage lifestyle is
already filled with so much competition that it would be \cloze
to plan activities in which there are more winners than losers,
\cloze , publishing newsletters with many student-written book
reviews, \cloze student artwork, and sponsoring book discussion
clubs. A variety of small clubs can provide \cloze opportunities
for leadership, as well as for practice in successful \cloze
dynamics. Making friends is extremely important to teenagers, and many
shy students need the \cloze of some kind of organization with a
supportive adult \cloze visible in the background.

In these activities, it is important to remember that the young teens
have \cloze attention spans. A variety of activities should be
organized \cloze participants can remain active as long as they
want and then go on to \cloze else without feeling guilty and
without letting the other participants \cloze. This does not
mean that adults must accept irresponsibility. \cloze they can
help students acquire a sense of commitment by \cloze for roles
that are within their \cloze and their attention spans and by
having clearly stated rules.


\newpage

\begin{enumerate}
	%\renewcommand{\labelenumi}{\arabic{enumi}.}
	% A(\Alph) a(\alph) I(\Roman) i(\roman) 1(\arabic)
	%设定全局标号series=example	%引用全局变量resume=example
	%[topsep=-0.3em,parsep=-0.3em,itemsep=-0.3em,partopsep=-0.3em]
	%可使用leftmargin调整列表环境左边的空白长度 [leftmargin=0em]
	\item
\fourchoices
{thought}
{idea}
{opinion}
{advice}




\item

\fourchoices
{strengthen}
{accommodate}
{stimulate}
{enhance}



\item


\fourchoices
{care}
{nutrition}
{exercise}
{leisure}




\item


\fourchoices
{If}
{Although}
{Whereas}
{Because}




\item

\fourchoices
{assistance}
{guidance}
{confidence}
{tolerance}



\item


\fourchoices
{claimed}
{admired}
{ignored}
{surpassed}




\item


\fourchoices
{improper}
{risky}
{fair}
{wise}




\item


\fourchoices
{in effect}
{as a result}
{for example}
{in a sense}





\item

\fourchoices
{displaying}
{describing}
{creating}
{exchanging}



\item


\fourchoices
{durable}
{excessive}
{surplus}
{multiple}




\item

\fourchoices
{group}
{individual}
{personnel}
{corporation}



\item


\fourchoices
{consent}
{insurance}
{admission}
{security}




\item

\fourchoices
{particularly}
{barely}
{definitely}
{rarely}



\item


\fourchoices
{similar}
{long}
{different}
{short}




\item


\fourchoices
{if only}
{now that}
{so that}
{even if}




\item

\fourchoices
{everything}
{anything}
{nothing}
{something}


\item


\fourchoices
{off}
{down}
{out}
{alone}




\item

\fourchoices
{On the contrary}
{On the average}
{On the whole}
{On the other hand}


\item


\fourchoices
{making}
{standing}
{planning}
{taking}




\item

\fourchoices
{capability}
{responsibility}
{proficiency}
{efficiency}

\end{enumerate}



\section{Reading Comprehension}

\noindent
\textbf{Part A}\\
\textbf{Directions:}\\
Read the following four texts. Answer the questions below each
	text by choosing A, B, C or
	D. Mark your answers
	on ANSWER SHEET 1. (40 points)

\newpage
\subsection{Text 1}


Wild Bill Donovan would have loved the Inter net. The American spymaster
who built the Office of Strategic Services in the World War Ⅱ and later
laid the roots for the CIA was fascinated with information. Donovan
believed in using whatever tools came to hand in the ``great game'' of
espionage---spying as a ``profession.'' These days the Net, which has
already re-made such everyday pastimes as buying books and sending mail,
is reshaping Donovan's vocation as well.

The latest revolution isn't simply a matter of gentlemen reading other
gentlemen's e-mail. That kind of electronic spying has been going on for
decades. In the past three or four years, the World Wide Web has given
birth to a whole industry of point-and-click spying. The spooks call it
``open source intelligence,'' and as the Net grows, it is becoming
increasingly influential. In 1995 the CIA held a contest to see who
could compile the most data about Burundi. The winner, by a large
margin, was a tiny Virginia company called Open-Source Solutions, whose
clear advantage was its mastery of the electronic world.

Among the firms making the biggest splash in the new world is
Straitford, Inc. , a private intelligence-analysis firm based in Austin,
Texas. Straitford makes money by selling the results of spying (covering
nations from Chile to Russia) to corporations like energy-services firm
McDermott International. Many of its predictions are available online at \emph{www.straiford.com}.

Straiford president George Friedman says he sees the online world as a
kind of mutually reinforcing tool for both information collection and
distribution, a spymaster's dream. Last week his firm was busy vacuuming
up data bits from the far corners of the world and predicting a crisis
in Ukraine. ``As soon as that report runs, we'll suddenly get 500 new
internet sign-ups from Ukraine,'' says Friedman, a former political
science professor. ``And we'll hear back from some of them.''
Open-source spying does have its risks, of course, since it can be
difficult to tell good information from bad. That's where Straitford
earns its keep.

Friedman relies on a lean staff of 20 in Austin. Several of his staff
members have military-intelligence backgrounds. He sees the firm's
outsider status as the key to its success. Straitford's briefs don't
sound like the usual Washington back-and-forthing, whereby agencies
avoid dramatic declarations on the chance they might be wrong.
Straitford, says Friedman, takes pride in its independent voice.


\begin{enumerate}[resume]
	%\renewcommand{\labelenumi}{\arabic{enumi}.}
	% A(\Alph) a(\alph) I(\Roman) i(\roman) 1(\arabic)
	%设定全局标号series=example	%引用全局变量resume=example
	%[topsep=-0.3em,parsep=-0.3em,itemsep=-0.3em,partopsep=-0.3em]
	%可使用leftmargin调整列表环境左边的空白长度 [leftmargin=0em]
	\item
The emergence of the Net has \lineread.


\fourchoices
{received support from fans like Donovan}
{remolded the intelligence services}
{restored many common pastimes}
{revived spying as a profession}


\item
 Donovan's story is mentioned in the text to \lineread.

\fourchoices
{introduce the topic of online spying}
{show how he fought for the US}
{give an episode of the information war}
{honor his unique services to the CIA}


\item
The phrase ``making the biggest splash'' (line 1, paragraph 3) most
probably means \lineread.


\fourchoices
{causing the biggest trouble}
{exerting the greatest effort}
{achieving the greatest success}
{enjoying the widest popularity}



\item
 It can be learned from paragraph 4 that \lineread.

\fourchoices
{straitford's prediction about Ukraine has proved true}
{straitford guarantees the truthfulness of its information}
{straitford's business is characterized by unpredictability}
{straitford is able to provide fairly reliable information}


\item
Straitford is most proud of its \lineread.


\fourchoices
{official status}
{nonconformist image}
{efficient staff}
{military background}



\end{enumerate}


\newpage
\subsection{Text 2}


To paraphrase 18th-century statesman Edmund Burke,
``all that is needed for the triumph of a misguided cause is that good
people do nothing.'' One such cause now seeks to end biomedical research
because of the theory that animals have rights ruling out their use in
research. Scientists need to respond forcefully to animal rights
advocates, whose arguments are confusing the public and thereby
threatening advances in health knowledge and care. Leaders of the animal
rights movement target biomedical research because it depends on public
funding, and few people understand the process of health care research.
Hearing allegations of cruelty to animals in research settings, many are
perplexed that anyone would deliberately harm an animal.

For example, a grandmotherly woman staffing an animal rights booth at a
recent street fair was distributing a brochure that encouraged readers
not to use anything that comes from or is tested in animals---no meat,
no fur, no medicines. Asked if she opposed immunizations, she wanted to
know if vaccines come from animal research. When assured that they do,
she replied, ``Then I would have to say yes.'' Asked what will happen
when epidemics return, she said, ``Don't worry, scientists will find
some way of using computers.'' Such well-meaning people just don't
understand.

Scientists must communicate their message to the public in a
compassionate, understandable way---in human terms, not in the language
of molecular biology. We need to make clear the connection between
animal research and a grandmother's hip replacement, a father's bypass
operation, a baby's vaccinations, and even a pet's shots. To those who
are unaware that animal research was needed to produce these treatments,
as well as new treatments and vaccines, animal research seems wasteful
at best and cruel at worst.

Much can be done. Scientists could ``adopt'' middle school classes and
present their own research. They should be quick to respond to letters
to the editor, lest animal rights misinformation go unchallenged and
acquire a deceptive appearance of truth. Research institutions could be
opened to tours, to show that laboratory animals receive humane care.
Finally, because the ultimate stakeholders are patients, the health
research community should actively recruit to its cause not only
well-known personalities such as Stephen Cooper, who has made courageous
statements about the value of animal research, but all who receive
medical treatment. If good people do nothing, there is a real
possibility that an uninformed citizenry will extinguish the precious
embers of medical progress.


\begin{enumerate}[resume]
	%\renewcommand{\labelenumi}{\arabic{enumi}.}
	% A(\Alph) a(\alph) I(\Roman) i(\roman) 1(\arabic)
	%设定全局标号series=example	%引用全局变量resume=example
	%[topsep=-0.3em,parsep=-0.3em,itemsep=-0.3em,partopsep=-0.3em]
	%可使用leftmargin调整列表环境左边的空白长度 [leftmargin=0em]
	\item
 The author begins his article with Edmund Burke's words to \lineread.


\fourchoices
{call on scientists to take some actions}
{criticize the misguided cause of animal rights}
{warn of the doom of biomedical research}
{show the triumph of the animal rights movement}


\item
Misled people tend to think that using an animal in research is \lineread.


\fourchoices
{cruel but natural}
{inhuman and unacceptable}
{inevitable but vicious}
{pointless and wasteful}


\item
 The example of the grandmotherly woman is used to show the public's \lineread.


\fourchoices
{discontent with animal research}
{ignorance about medical science}
{indifference to epidemics}
{anxiety about animal rights}


\item
The author believes that, in face of the challenge from animal
rights advocates, scientists should \lineread.


\fourchoices
{communicate more with the public}
{employ hi-tech means in research}
{feel no shame for their cause}
{strive to develop new cures}


\item
From the text we learn that Stephen Cooper is \lineread.


\fourchoices
{a well-known humanist}
{a medical practitioner}
{an enthusiast in animal rights}
{a supporter of animal research}


\end{enumerate}



\newpage
\subsection{Text 3}


In recent years, railroads have been combining with each other, merging
into supersystems, causing heightened concerns about monopoly. As
recently as 1995, the top four railroads accounted for under 70 percent
of the total ton-miles moved by rails. Next year, after a series of
mergers is completed, just four railroads will control well over 90
percent of all the freight moved by major rail carriers.

Supporters of the new supersystems argue that these mergers will allow
for substantial cost reductions and better coordinated service. Any
threat of monopoly, they argue, is removed by fierce competition from
trucks. But many shippers complain that for heavy bulk commodities
traveling long distances, such as coal, chemicals, and grain, trucking
is too costly and the railroads therefore have them by the throat.

The vast consolidation within the rail industry means that most shippers
are served by only one rail company. Railroads typically charge
such``captive''shippers 20 to 30 percent more than they do when another
railroad is competing for the business. Shippers who feel they are being
overcharged have the right to appeal to the federal government's Surface
Transportation Board for rate relief, but the process is expensive, time
consuming, and will work only in truly extreme cases.

Railroads justify rate discrimination against captive shippers on the
grounds that in the long run it reduces everyone's cost. If railroads
charged all customers the same average rate, they argue, shippers who
have the option of switching to trucks or other forms of transportation
would do so, leaving remaining customers to shoulder the cost of keeping
up the line. It's theory to which many economists subscribe, but in
practice it often leaves railroads in the position of determining which
companies will flourish and which will fail. ``Do we really want
railroads to be the \uline{arbiters} of who wins and who loses in the
marketplace?''asks Martin Bercovici, a Washington lawyer who frequently
represents shipper.

Many captive shippers also worry they will soon be hit with a round of
huge rate increases. The railroad industry as a whole, despite its
brightening fortuning fortunes, still does not earn enough to cover the
cost of the capital it must invest to keep up with its surging traffic.
Yet railroads continue to borrow billions to acquire one another, with
Wall Street cheering them on. Consider the \$10.2 billion bid by Norfolk
Southern and CSX to acquire Conrail this year. Conrail's net railway
operating income in 1996 was just \$427 million, less than half of the
carrying costs of the transaction. Who's going to pay for the rest of
the bill? Many captive shippers fear that they will, as Norfolk Southern
and CSX increase their grip on the market.


\begin{enumerate}[resume]
	%\renewcommand{\labelenumi}{\arabic{enumi}.}
	% A(\Alph) a(\alph) I(\Roman) i(\roman) 1(\arabic)
	%设定全局标号series=example	%引用全局变量resume=example
	%[topsep=-0.3em,parsep=-0.3em,itemsep=-0.3em,partopsep=-0.3em]
	%可使用leftmargin调整列表环境左边的空白长度 [leftmargin=0em]
	\item
According to those who support mergers, railway monopoly is unlikely
because \lineread.


\fourchoices
{cost reduction is based on competition}
{services call for cross-trade coordination}
{outside competitors will continue to exist}
{shippers will have the railway by the throat}


\item
What is many captive shippers' attitude towards the consolidation in
the rail industry?


\fourchoices
{Indifferent.}
{Supportive.}
{Indignant.}
{Apprehensive.}


\item
 It can be inferred from paragraph 3 that \lineread.


\fourchoices
{shippers will be charged less without a rival railroad}
{there will soon be only one railroad company nationwide}
{overcharged shippers are unlikely to appeal for rate relief}
{a government board ensures fair play in railway business}

\item
The word ``arbiters''(line 7, paragraph 4) most probably refers to
those \lineread.


\fourchoices
{who work as coordinators}
{who function as judges}
{who supervise transactions}
{who determine the price}


\item
According to the text, the cost increase in the rail industry is
mainly caused by \lineread.


\fourchoices
{the continuing acquisition}
{the growing traffic}
{the cheering Wall Street}
{the shrinking market}


\end{enumerate}


\newpage
\subsection{Text 4}


It is said that in England death is pressing, in Canada inevitable and
in California optional. Small wonder. Americans' life expectancy has
nearly doubled over the past century. Failing hips can be replaced,
clinical depression controlled, cataracts removed in a 30-minute
surgical procedure. Such advances offer the aging population a quality
of life that was unimaginable when I entered medicine 50 years ago. But
not even a great health-care system can cure death---and our failure to
confront that reality now threatens this greatness of ours.

Death is normal; we are genetically programmed to disintegrate and
perish, even under ideal conditions. We all understand that at some
level, yet as medical consumers we treat death as a problem to be
solved. Shielded by third-party payers from the cost of our care, we
demand everything that can possibly be done for us, even if it's
useless. The most obvious example is late-stage cancer care.
Physicians---frustrated by their inability to cure the disease and
fearing loss of hope in the patient---too often offer aggressive
treatment far beyond what is scientifically justified.

In 1950, the US spent \$12.7 billion on health care. In 2002, the cost
will be \$1,540 billion. Anyone can see this trend is unsustainable. Yet
few seem willing to try to reverse it. Some scholars conclude that a
government with finite resources should simply stop paying for medical
care that sustains life beyond a certain age---say 83 or so. Former
Colorado governor Richard Lamm has been quoted as saying that the old
and infirm ``have a duty to die and get out of the way'', so that
younger, healthier people can realize their potential.

I would not go that far. Energetic people now routinely work through
their 60s and beyond, and remain dazzlingly productive. At 78, Viacom
chairman Sumner Redstone jokingly claims to be 53. Supreme Court Justice
Sandra Day O'Connor is in her 70 s, and former surgeon general
C. Everett
Koop chairs an Internet start-up in his 80 s. These leaders are living
proof that prevention works and that we can manage the health problems
that come naturally with age. As a mere 68-year-old, I wish to age as
productively as they have.

Yet there are limits to what a society can spend in this pursuit. As a
physician, I know the most costly and dramatic measures may be
ineffective and painful. I also know that people in Japan and Sweden,
countries that spend far less on medical care, have achieved longer,
healthier lives than we have. As a nation, we may be overfunding the
quest for unlikely cures while underfunding research on humbler
therapies that could improve people's lives.


\begin{enumerate}[resume]
	%\renewcommand{\labelenumi}{\arabic{enumi}.}
	% A(\Alph) a(\alph) I(\Roman) i(\roman) 1(\arabic)
	%设定全局标号series=example	%引用全局变量resume=example
	%[topsep=-0.3em,parsep=-0.3em,itemsep=-0.3em,partopsep=-0.3em]
	%可使用leftmargin调整列表环境左边的空白长度 [leftmargin=0em]
	\item
 What is implied in the first sentence?

\fourchoices
{Americans are better prepared for death than other people.}
{Americans enjoy a higher life quality than ever before.}
{Americans are over-confident of their medical technology.}
{Americans take a vain pride in their long life expectancy.}


\item
The author uses the example of caner patients to show that \lineread.


\fourchoices
{medical resources are often wasted}
{doctors are helpless against fatal diseases}
{some treatments are too aggressive}
{medical costs are becoming unaffordable}


\item
The author's attitude toward Richard Lamm's remark is one of \lineread.


\fourchoices
{strong disapproval}
{reserved consent}
{slight contempt}
{enthusiastic support}



\item
 In contras to the US, Japan and Sweden are funding their medical
care \lineread.


\fourchoices
{more flexibly}
{more extravagantly}
{more cautiously}
{more reasonably}


\item
The text intends to express the idea that \lineread.


\fourchoices
{medicine will further prolong people's lives}
{life beyond a certain limit is not worth living}
{death should be accepted as a fact of life}
{excessive demands increase the cost of health care}


\end{enumerate}



\newpage

\noindent
\textbf{Part B}\\
\textbf{Directions:}\\
{Read the following text carefully and then translate the
	underlined segments into Chinese. Your translation should be written
	clearly on ANSWER SHEET 2. (10 points)


\TiGanSpace



Human beings in all times and places think about their world and wonder
at their place in it. Humans are thoughtful and creative, possessed of
insatiable curiosity. \transnum \uline{Furthermore, humans have the
	ability to modify the environment in which they live, thus subjecting
	all other life forms to their own peculiar ideas and fancies.}
Therefore, it is important to study humans in all their richness and
diversity in a calm and systematic manner, with the hope that the
knowledge resulting from such studies can lead humans to a more
harmonious way of living with themselves and with all other life forms
on this planet Earth.

``Anthropology'' derives from the Greek words \emph{anthropos} ``human''
and \emph{logos} ``the study of.'' By its very name, anthropology
encompasses the study of all humankind.

Anthropology is one of the social sciences. \transnum \uline{Social
	science is that branch of intellectual enquiry which seeks to study
	humans and their endeavors in the same reasoned, orderly, systematic,
	and dispassioned manner that natural scientists use for the study of
	natural phenomena}.

Social science disciplines include geography, economics, political,
science, psychology, and sociology. Each of these social sciences has a
subfield or specialization which lies particularly close to
anthropology.

All the social sciences focus upon the study of humanity. Anthropology
is a field-study oriented discipline which makes extensive use of the
comparative method in analysis. \transnum \uline{The emphasis on data
	gathered first-hand, combined with a cross-cultural perspective brought
	to the analysis of cultures past and present, makes this study a unique
	and distinctly important social science}.

Anthropological analyses rest heavily upon the concept of culture. Sir
Edward Tylor's formulation of the concept of culture was one of the
great intellectual achievements of 19th century
science. \transnum \uline{Tylor defined culture as ``\ldots that complex
	whole which includes belief, art, morals, law, custom, and any other
	capabilities and habits acquired by man as a member of society.''} This
insight, so profound in its simplicity, opened up an entirely new way of
perceiving and understanding human life. Implicit within Tylor's
definition is the concept that culture is learned. shared, and patterned
behavior.

 \transnum \uline{Thus, the anthropological concept of ``culture,'' like
	the concept of ``set'' in mathematics, is an abstract concept which
	makes possible immense amounts of concrete research and understanding.}



\section{Writing}




\noindent
\textbf{46. Directions}

Study the following set of drawings carefully and write an essay
entitled in which you should
\begin{listwrite}
\item 
describe the set of drawings, interpret its meaning, and

\item 
point out its implications in our life.
\end{listwrite}


You should write about 200 words neatly on ANSWER SHEET 2. (20 points)


\begin{figure}[h!]
	\centering
	\includesvg[width=0.67\linewidth]{picture/svg/picture-02}
	\caption*{温室花朵经不起风雨}
\end{figure}




